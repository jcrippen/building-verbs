%!TEX root = ../building-verbs.tex
%!TEX encoding = UTF-8 Unicode
%%
%% Font support for special symbols (e.g. math).
%%

% https://tex.stackexchange.com/questions/367773/command-textsuperscript-unavailable-in-encoding-tu?noredirect=1
\DeclareTextSymbol{\texttilde} {TU} {"007E}

%% Short command for a superscript question mark.
\ProvideDocumentCommand \supq {} {\textsuperscript*{?}}

%% Load the newunicodechar package.
%%
%% This package lets you define arbitrary macros to run for particular
%% Unicode characters (or any character, really). This allows you to
%% switch fonts per character, and thus use different fonts for e.g.
%% mathematical symbols, Hebrew letters, etc. It should not be used
%% for long stretches of text, which instead should be done with a
%% specialized command or environment that changes the font in a
%% more general way.
%%
%% All that the \newunicodechar command is really doing under the
%% hood is something like plain TeX
%%   \catcode`\X=\active
%%   \defX{\textbf{\char"2323}}
%% but it's much easier to use and so less error-prone.
\usepackage{newunicodechar}

%% This font is specifically for the check symbol that indicates
%% grammaticality or felicity. The macro can be used in contexts
%% where the character can't, with the same printed result. E.g.
%% ExPex can use \ck to determine judgement width, but not ✓.
%%
%% NOTE: You need two braces here because of macro processing.
%% You can also use \begingroup ... \endgroup. It's a TeX thing.
\ProvideDocumentCommand \ck {} {{\ckfont ✓}}
%% Two braces again, here and elsewhere below.
\newunicodechar{✓}{{\ck}}

%% This font is for the X symbol that contrasts with the check above.
%% The same principles apply.
%\newfontfamily{\badxfont}[Scale=1]{Zapf Dingbats}
\newfontfamily{\badxfont}{Zapf Dingbats}
\ProvideDocumentCommand \badx {} {{\badxfont ✓}}
\newunicodechar{✗}{{\badxfont ✗}}

%% The slashed zero as the null symbol.
%% The macro is defined in crippenmacros.sty.
\newunicodechar{∅}{{\nul}}

%% The Brill font has Greek-style default versions of {β, θ, λ, χ} and
%% Roman-style versions available with ss20. But I only want
%% the roman θ and not the others. Brill 2.02 doesn’t implement
%% character variant features (cvXX) for these, so the only way to access
%% them is through ss20. I use newunicodechar to automatically add
%% ss20 whenever θ appears.
%%
%% This solution completely ignores the current font. That means that ss20 will
%% be applied in every situation, no matter what. So even if the font is no
%% longer Brill for some reason, θ will still show up with ss20. It works fine if
%% you’re consistent and careful, but it isn’t ideal. cvXX would be ideal.
\newunicodechar{θ}{{\addfontfeature{StylisticSet=20}θ}}

%% Mathematical symbols.
%%
%% If you have things with diacritics (e.g. math negation) then if
%% these are decomposed (NFD) it will make TeX barf. YHBW.
%%
%% The double braces are on purpose, with the inner pair
%% restricting the scope of the font switching commands.
%%
%% FIXME: This should be in an array with a loop instantiating each
%% character, rather than repeating the same code over and over again.
%% It wouldn't make it any more efficient, but it'd be easier to modify.
%% Actually it would be a couple array+loops, one for each font.

%% Set stuff.
\newunicodechar{∈}{{\mathemafont ∈}}	% element
\newunicodechar{∉}{{\mathemafont ∉}}	% not element
\newunicodechar{⊂}{{\mathemafont ⊂}}	% proper subset
\newunicodechar{⊃}{{\mathemafont ⊃}}	% proper superset
\newunicodechar{⊆}{{\mathemafont ⊆}}	% total subset
\newunicodechar{⊇}{{\mathemafont ⊇}}	% total superset
\newunicodechar{⊊}{{\mathemafont ⊊}}	% nontotal subset
\newunicodechar{⊋}{{\mathemafont ⊋}}	% nontotal superset
\newunicodechar{⊈}{{\mathemafont ⊈}}	% not total subset
\newunicodechar{⊉}{{\mathemafont ⊉}}	% not total superset
\newunicodechar{∪}{{\mathemafont ∪}}	% union
\newunicodechar{∩}{{\mathemafont ∩}}	% intersection
\newunicodechar{⋃}{{\mathemafont ⋃}}	% big union

%% Logic and quantification.
\newunicodechar{≡}{{\mathemafont ≡}}	% triple equals
\newunicodechar{≢}{{\mathemafont ≢}}	% triple not equals
\newunicodechar{≅}{{\mathemafont ≅}}	% triple approximately equals
\newunicodechar{∧}{{\mathemafont ∧}}	% logical and
\newunicodechar{∨}{{\mathemafont ∨}}	% logical or
\newunicodechar{∀}{{\mathemafont ∀}}	% for all
\newunicodechar{∃}{{\mathemafont ∃}}	% exists
\newunicodechar{∄}{{\mathemafont ∄}}	% not exists

%% Orderings.
\newunicodechar{⊤}{{\mathemafont ⊤}}	% supremum, top
\newunicodechar{⊥}{{\mathemafont ⊥}}	% infimum, bottom
\newunicodechar{⊑}{{\mathemafont ⊑}}	% precedes
\newunicodechar{⊒}{{\mathemafont ⊒}}	% succedes
\newunicodechar{⊏}{{\mathemafont ⊏}}	% strictly precedes
\newunicodechar{⊐}{{\mathemafont ⊐}}	% strictly succedes
\newunicodechar{⋢}{{\mathemafont ⋢}}	% not succeeds
\newunicodechar{≪}{{\mathemafont ≪}}	% much less than
\newunicodechar{≫}{{\mathemafont ≫}}	% much greater than
\newunicodechar{≶}{{\mathemafont ≶}}	% less than over greater than
\newunicodechar{≷}{{\mathemafont ≷}}	% greater than over less than
\newunicodechar{≸}{{\mathemafont ≹}}	% neither less than nor greater than
\newunicodechar{≹}{{\mathemafont ≹}}	% neither greater than nor less than
\newunicodechar{≼}{{\mathemafont ≼}}	% curved precedes
\newunicodechar{≽}{{\mathemafont ≽}}	% curved succedes
\newunicodechar{≺}{{\mathemafont ≺}}	% curved strictly precedes
\newunicodechar{≻}{{\mathemafont ≻}}	% curved strictly succedes
\newunicodechar{⊀}{{\mathemafont ⊀}}	% not curved strictly succeeds

%% Arrows.
\newunicodechar{⇒}{{\mathemafont ⇒}} % double east arrow
\newunicodechar{⇐}{{\mathemafont ⇐}} % double west arrow
\newunicodechar{⇔}{{\mathemafont ⇔}} % double east-west arrow
\newunicodechar{⇏}{{\mathemafont ⇏}} % not double east arrow
\newunicodechar{⇍}{{\mathemafont ⇍}} % not double west arrow
\newunicodechar{⇎}{{\mathemafont ⇎}} % not double east-west arrow
\newunicodechar{↦}{{\mathemafont ↦}} % maps to
\newunicodechar{↤}{{\mathemafont ↤}} % maps from
\newunicodechar{⤵}{{\mathemafont \raisebox{-0.5ex}{⤵}}} % right curve downwards arrow

%% Calligraphic script letters.
\newunicodechar{𝒞}{{\mathecalfont 𝒞}}	% script capital C
\newunicodechar{ℰ}{{\mathecalfont ℰ}}	% script capital E
\newunicodechar{𝒬}{{\mathecalfont 𝒬}}	% script capital Q
\newunicodechar{ℛ}{{\mathecalfont ℛ}}	% script capital R

%% Phonetic stuff.
\newunicodechar{ᵒ}{{\phonfont ᵒ}}	% superscript o
\newunicodechar{ᵏ}{{\phonfont ᵏ}}	% superscript k
\newunicodechar{ᵘ}{{\phonfont ᵘ}}	% superscript u
\newunicodechar{ᵉ}{{\phonfont ᵉ}}	% superscript e
\newunicodechar{ᵗ}{{\phonfont ᵗ}}	% superscript t
\newunicodechar{ꜝ}{{\phonfont ꜝ}}	% superscript !
\newunicodechar{ꜜ}{{\phonfont ꜜ}}	% superscript down arrow
\newunicodechar{ꜛ}{{\phonfont ꜛ}}	% superscript up arrow
\newunicodechar{ₚ}{{\phonfont ₚ}}	% subscript p
\newunicodechar{ᴴ}{{\phonfont ᴴ}}	% superscript capital H

%% Miscellanea.
%\newunicodechar{𝑣}{{\mathemafont 𝑣}}
\newunicodechar{¬}{{\mathemafont ¬}}	% not: Brill has this but it’s too low
\newunicodechar{⋮}{{\mathemafont ⋮}}	% vertical ellipsis
\newunicodechar{ℓ}{{\mathemafont ℓ}}	% script small l
\newunicodechar{⊗}{{\upshape\mathemafont ⊗}}	% circle times
\newunicodechar{⬚}{{\symfont ⬚}}	% dotted box