%!TEX root = ../building-verbs.tex
%%
%% Indexing and glossary support.
%%

%% Indicate calls to index commands in the typeset output.
%%
%% NOTE: This package must be loaded before imakeidx.
\usepackage{showidx}

%% Modify the \showindexmarks formatting.
\indexmarkstyle{\normalfont\miniscule\ttfamily}

%% Build indices concurrently with typesetting using \write18.
\usepackage[noautomatic]{imakeidx}

%%%%
%%%% Index configuration.
%%%%

%% Index declarations.
\makeindex[name=subjects, title={Index of subjects}, columns=2]
\makeindex[name=places, title={Index of placenames}, columns=2]
\makeindex[name=people, title={Index of personal names}, columns=2]
\makeindex[name=languages, title={Index of languages}, columns=2, program=texindy]
\makeindex[name=research, title={Index of research topics}, columns=1]

%% Mid-level commands for adding index entries.
%%
%% The low level command is \index but it's too primitive, so we replace it
%% with some specialized commands here. There should be one command
%% defined below for each index declared above.
\ProvideDocumentCommand \subjectindex { >{\TrimSpaces}m } {\index[subjects]{#1}}
\ProvideDocumentCommand \placeindex { >{\TrimSpaces}m } {\index[places]{#1}}
\ProvideDocumentCommand \peopleindex { >{\TrimSpaces}m } {\index[people]{#1}}
\ProvideDocumentCommand \languageindex { >{\TrimSpaces}m } {\index[languages]{#1}}
\ProvideDocumentCommand \researchindex { >{\TrimSpaces}m } {\index[research]{#1}}
%% These are equivalents but add |indexprimary to emphasize the page number.
\ProvideDocumentCommand \subjectindexp { >{\TrimSpaces}m } {\index[subjects]{#1|indexprimary}}
\ProvideDocumentCommand \placeindexp { >{\TrimSpaces}m } {\index[places]{#1|indexprimary}}
\ProvideDocumentCommand \peopleindexp { >{\TrimSpaces}m } {\index[people]{#1|indexprimary}}
\ProvideDocumentCommand \languageindexp { >{\TrimSpaces}m } {\index[languages]{#1|indexprimary}}
\ProvideDocumentCommand \researchindexp { >{\TrimSpaces}m } {\index[research]{#1|indexprimary}}


%% High-level commands for adding index entries.
%%
%% These commands take a list of index entries and add them to the appropriate
%% index. They are suitable for use in the body where an index entry is not otherwise
%% added by some formatting or data handling command like \term.
\ProvideDocumentCommand \SubjectIndex { >{\SplitList{;}}m } {\ProcessList{#1}{\subjectindex}}
\ProvideDocumentCommand \PlaceIndex { >{\SplitList{;}}m } {\ProcessList{#1}{\placeindex}}
\ProvideDocumentCommand \PeopleIndex { >{\SplitList{;}}m } {\ProcessList{#1}{\peopleindex}}
\ProvideDocumentCommand \LanguageIndex { >{\SplitList{;}}m } {\ProcessList{#1}{\languageindex}}
\ProvideDocumentCommand \ResearchIndex { >{\SplitList{;}}m } {\ProcessList{#1}{\researchindex}}

%%%%
%%%% Index entry commands.
%%%%

%% Terminology definitions.
%%
%% This prints a newly defined term in small capitals and adds it to the index.
%% The optional argument can contain alternative index entries to replace the term.
%%
%% Usage:
%%   We argue that \term[index entry; entry!index]{index entries} are defined as …

%% The user interface.
%% This has to be \Declare rather than \Provide because I have defined \term
%% elsewhere and that definition will conflict with this one.
\DeclareDocumentCommand \term { >{\SplitList{;}}o m }
	{\IfNoValueOrEmptyTF{#1}
		{\subjectindexp{#2}\termformat{#2}}
		{\ProcessList{#1}{\subjectindexp}\termformat{#2}}}

%% The formatting command.
\ProvideDocumentCommand \termformat { m } {\textsc{#1}}


%% Place name mentions.
%%
%% This prints a place name and adds it to the index. The optional argument
%% contains a list of other names to add to the index.

%% The user interface.
\ProvideDocumentCommand \placename { >{\SplitList{;}}o m }
	{\IfNoValueOrEmptyTF{#1}
		{\languageindex{#2}\placenameformat{#2}}
		{\ProcessList{#1}{\placeindex}\placenameformat{#2}}}

%% The formatting command.
\ProvideDocumentCommand \placenameformat { m } {#1}


%% Language name mentions.
%%
%% This prints a language name and adds it to the index. The optional argument
%% contains a list of language names to add to the index.
%%
%% Usage:
%%  The \langname[Dakelh; Carrier]{Dakelh} language also exhibits…

%% The user interface.
\ProvideDocumentCommand \langname { >{\SplitList{;}}o m }
	{\IfNoValueOrEmptyTF{#1}
		{\languageindex{#2}\langformat{#2}}
		{\ProcessList{#1}{\languageindex}\langformat{#2}}}

%% The formatting command.
\ProvideDocumentCommand \langformat { m } {#1}



%% Personal name (people) mentions.
%%
%%   arg 1: star; wraps everything in parentheses
%%     useful when used as a citation in an example
%%   arg 2: optional Tlingit name
%%   arg 3: first name
%%   arg 4: optional abbreviation to print for first name
%%   arg 4: last name
%
%% If the star is present then the whole thing is printed in parentheses.
%% The optional argument is a Tlingit name which is entered separately
%% into the index as well as included in the other index entry.
\ProvideDocumentCommand \personname { s d() m d() m }
	{\IfBooleanTF{#1}{(}{}%
	\IfNoValueOrEmptyTF{#2}{}{\textit{#2}\PeopleIndex{#2}\space}%	Tlingit name
	\IfNoValueOrEmptyTF{#4}{#3}{#4}\space%	first name
	#5%	last name
	\PeopleIndex{#5, #3\IfNoValueOrEmptyTF{#2}{}{ \textit{#2}}}%	index with optional Tlingit name
	\IfBooleanTF{#1}{)}{}}

%% Tlingit-only personal name.
%%
%% A simpler macro for Tlingit-only names, e.g. from Swanton.
\ProvideDocumentCommand \tlingitname { s m }
	{\IfBooleanTF{#1}{#2}{\fm{#2}}\PeopleIndex{#2}}

%% Clan name.
%%
%% A special macro that includes “(clan)” in the index entry.
%%
%% arg 1: star, wraps everything in parentheses
%% arg 2: clan name
\ProvideDocumentCommand \clanname { s m }
	{\IfBooleanTF{#1}{(}{}%
	\fm{#2}%
	\PeopleIndex{#2 (clan)}%
	\IfBooleanTF{#1}{)}{}}


%%%%
%%%% Index formatting.
%%%%

%% Highlight command to indicate primary reference for index entry.
\ProvideDocumentCommand \indexprimary { m } {\textbf{#1}}