%!TEX root = ../building-verbs.tex
%%
%% Miscellaneous macrology.
%%

%% This is a general collection of personal macros.
%% Several things below are derived from this package.
%% The date ensures that any versions before that date
%% will trigger a warning.
\usepackage{crippenmacros}[2017/04/24]

%% Set up TikZ example arrows.
\tikzexstylesetup
\tikzset{exarrows/.style={semithick, arrows = {-Stealth[scale=0.75]}}}
\tikzset{exarrowsagr/.style={thin, double, arrows = {-Stealth[width=6pt, open, scale=1]}}}

%% Move arrow style.
\tikzset{mvarrows/.style={semithick, arrows = -{Stealth[scale=1, scale length=1, scale width=1}}}

%% Agree arrow style.
\tikzset{agrarrows/.style={semithick, arrows = {Circle[open]}-{Circle[open]}}}

%% Tree style.
\tikzset{mytree/.style={align=center,
		anchor=north,
		parent anchor=south,
		child anchor=north,
		level distance=2em,
		sibling distance=4em}}

%% Interrupted tree branch style for 2em level distance.
\tikzset{dashbranch2em/.style={dash pattern={on 0.75em
										off 2pt on 2pt off 2pt on 2pt off 2pt on 2pt off 2pt
									on 0.75em}}}

%% Settings for the \gm gloss emphasis macro.
%%
%% Bold text.
\gmboldoff
%% Foreground colour.
\gmcoloroff
%% Background shading.
\gmbgcoloron
%% Adjust the two colours.
\gmcolorset{0.66,0,0}
\gmbgcolorset{0.9,0.9,0.9}

%% The Brill font doesn’t need its roots kerned.
\setlength{\rtkern}{0pt}
%% It also needs a wider underscore.
\setlength{\uscorewidth}{0.875em}

%% Use real double brackets for fixme comments.
\renewcommand*{\dblbrackleft}{⟦}
\renewcommand*{\dblbrackright}{⟧}

%% This is a small strut for table row spacing. It ensures enough space for
%% superpositioned and subpositioned diacritics, but doesn't extend as
%% much as a regular \strut. It has zero width like \strut.
\ProvideDocumentCommand \smstrut {} {\vphantom{V̈́V̨}}

%% Phantom (invisible) hyphens for alignment in examples.
\ProvideDocumentCommand \· {} {\phantom{-}}

%% Phantom (invisible) equals signs for alignment in examples.
\ProvideDocumentCommand \• {} {\phantom{=}}

%% Stigmata macros.
\ProvideDocumentCommand \† {} {\llap{\textsuperscript*{†}}}

%% A feature.
\ProvideDocumentCommand \ftr { m } {[#1]}

%% Command to force lining numerals instead of old style.
\ProvideDocumentCommand \liningnumerals {m}
	{\begingroup\addfontfeatures{Letters=Uppercase,Numbers=Lining}#1\endgroup}

%% Second language abbreviation.
\ProvideDocumentCommand \Lone {} {\liningnumerals{L1}}
\ProvideDocumentCommand \Ltwo {} {\liningnumerals{L2}}

%% Orthography abbreviation.
\ProvideDocumentCommand \NSone {} {\liningnumerals{NS1}}
\ProvideDocumentCommand \NStwo {} {\liningnumerals{NS2}}

%% Fundamental frequency.
\ProvideDocumentCommand \Fzero {} {\textit{f₀}}

%% Root as a symbol.
%%
%% This smashes the root into a 1em vertical box, adding slight whitespace on the right.
\ProvideDocumentCommand \rtsym {} {\raisebox{0pt}[0.75em][0.25em]{√\!}}

%% Empty spacing for gloss abbreviation (\glc) line.
\DeclareDocumentCommand {\glcsp} {s}
	{\hspace{1em}\IfBooleanTF{#1}{\hspace{1em}}}

%% Fake spacing for gloss text line (\gla), using ExPex’s length.
\DeclareDocumentCommand {\glasp} {} {\hspace\lingglspace}

%% Verb ephemera.
\ProvideDocumentCommand \vbeph { m m }
	{(\IfNoValueOrEmptyTF{#1}{}{\fm{#1}\IfNoValueOrEmptyTF{#2}{}{; }}\IfNoValueOrEmptyTF{#2}{}{#2})}

%% Verb lexical entry.
\ProvideDocumentCommand \vblex { m m m m }
	{\IfNoValueOrEmptyTF{#1}{}{\fm{#1} }\vbeph{#2}{#3}\IfNoValueOrEmptyTF{#4}{}{ ‘#4’}}

%% Verb derivation.
\ProvideDocumentCommand \vbderiv { m m m m }
	{\vblex{#1}{#2}{#3}{#4}}

%% Verb lexical entry in examples.
\ProvideDocumentCommand \exvblex { s m m m m }
	{\IfBooleanTF{#1}{}{\newline}\footnotesize\vblex{#2}{#3}{#4}{#5}\normalsize}

%% Verb derivation in examples.
\ProvideDocumentCommand \exvbderiv { s m m m m }
	{\IfBooleanTF{#1}{}{\newline}\footnotesize\vbderiv{#2}{#3}{#4}{#5}\normalsize}

%% Binominal species names.
\ProvideDocumentCommand \species { s m m o} 
	{\textit{#2}~\textit{#3}\IfNoValueOrEmptyTF{#4}{}{~\IfNoValueTF{#1}{(}{}#4\IfNoValueTF{#1}{)}{}}}

%% Some memoir configuration for marginal notes.
\marginparmargin{right}
\setmarginnotes{1ex}{4em}{0.5ex}
%% Margin note for example number.
\DeclareDocumentCommand {\exmn} {m} {\marginpar{\footnotesize{}(#1)}}

%% Terminology macro.
\DeclareDocumentCommand {\term} {m} {\sidepar{\footnotesize\textsf{#1}}\textbf{#1}}