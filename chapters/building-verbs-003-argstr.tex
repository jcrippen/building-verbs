%!TEX root = ../building-verbs.tex
%%
%% 3. Argument structure.
%%

\resetexcnt
\chapter{Argument structure}\label{ch:argstr}

\startcontents[chapters]
\noindent\rule[0.5em]{\textwidth}{\heavyrulewidth}
\printcontents[chapters]{}{1}{\setcounter{tocdepth}{2}}
\noindent\rule{\textwidth}{\heavyrulewidth}
\vspace{1\baselineskip}

%\clearpage

We discussed specific kinds of arguments – subject, object, etc.\ – in chapter \ref{ch:args}.
In this chapter we look at \term{argument structure} which is the pattern of arguments required by a verb.
The argument structure of a verb indicates the number of arguments that the verb requires, the syntactic kinds of arguments required, and the thematic roles applied to each argument \parencite{levin:2018}.
As detailed in chapter \ref{ch:args}, one particular kind of argument can be associated with several different thematic roles.
Since thematic roles are semantic and therefore harder to diagnose, we again focus on the syntactic properties of argument structure and only secondarily mention their associated thematic roles.

The simplest argument structure property is the number of arguments required by a given verb.
Transitive verbs (sec.\ \ref{sec:argstr-trans}) are verbs that require two arguments: a subject and an object.
Intransitive verbs (sec.\ \ref{sec:argstr-intrans}) are verbs that require one argument.
In Tlingit the single argument of an intransitive verb may be either an object (sec.\ \ref{sec:argstr-intrans-obj}) or a subject (sec.\ \ref{sec:argstr-intrans-subj}).
The transitive versus intransitive distinction covers the core arguments of subject and object.
But as noted in chapter \ref{ch:args} there are other kinds of arguments beyond the core arguments, namely obliques.
These oblique arguments give rise to other patterns of argument structure involving postposition phrases (PPs) and adverbs (sec.\ \ref{sec:argstr-other}).
The obliques extend the patterns of core argument structure to cover phenomena like applicatives (sec.\ \ref{sec:argstr-other-appl}), locations and paths (sec.\ \ref{sec:argstr-other-locpath}), comparisons (sec.\ \ref{sec:argstr-other-cmpv}), and manners (sec.\ \ref{sec:argstr-other-manner}).

\section{Transitives}\label{sec:argstr-trans}


\section{Intransitives}\label{sec:argstr-intrans}

\subsection{Object intransitives}\label{sec:argstr-intrans-obj}

\subsection{Subject intransitives}\label{sec:argstr-intrans-subj}

\section{Other patterns of argument structure}\label{sec:argstr-other}

\subsection{Applicative arguments}\label{sec:argstr-other-appl}

\subsection{Location and path arguments}\label{sec:argstr-other-locpath}

\subsection{Comparative arguments}\label{sec:argstr-other-cmpv}

\subsection{Manner arguments}\label{sec:argstr-other-manner}

\stopcontents[chapters]