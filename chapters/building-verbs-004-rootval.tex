%!TEX root = ../building-verbs.tex
%%
%% 4. Root valency.
%%

\resetexcnt
\chapter{Root valency}\label{ch:rootval}

\startcontents[chapters]
\noindent\rule[0.5em]{\textwidth}{\heavyrulewidth}
\printcontents[chapters]{}{1}{\setcounter{tocdepth}{2}}
\noindent\rule{\textwidth}{\heavyrulewidth}
\vspace{1\baselineskip}

%\clearpage

\section{Bivalent roots}\label{sec:rootval-bi}

\section{Monovalent roots}\label{sec:rootval-mono}

\section{Nullivalent roots}\label{sec:rootval-null}

\section{Other patterns of root valency}\label{sec:rootval-other}

If all we needed to know about root valency was the distinction between bivalent roots, monovalent roots, and nullivalent roots then the world of Tlingit root valency would be a simple place.
Unfortunately things are not so simple: there are roots which show patterns outside of this three-way categorization of root valency.

\stopcontents[chapters]