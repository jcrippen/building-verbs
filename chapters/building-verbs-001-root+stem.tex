%!TEX root = ../building-verbs.tex
%%
%% 1. Introduction.
%%

\resetexcnt
\chapter{Roots and stems}\label{ch:root+stem}

\startcontents[chapters]
\noindent\rule[0.5em]{\textwidth}{\heavyrulewidth}
\printcontents[chapters]{}{1}{\setcounter{tocdepth}{2}}
\noindent\rule{\textwidth}{\heavyrulewidth}
\vspace{1\baselineskip}

%\clearpage

\section{Roots}\label{sec:root+stem-roots}

All verbs in Tlingit are based on roots.
А root minimally encodes a sequence of sounds and a set of related meanings, though most roots include more than just these two kinds of information.
A typical example is the root \fm{\rt[¹]{ta}} ‘sg.\ sleep’.
In terms of sounds, \fm{\rt[¹]{ta}} encodes a sequence of the aspirated alveolar stop \fm{t} /\ipa{tʰ}/ followed by the low vowel \fm{a} /\ipa{a}/.
In terms of meaning, \fm{\rt[¹]{ta}} encodes ‘sleep’ and thus the idea of physical rest with a reduced state of consciousness.
This root also encodes other information which will be discussed later.

All vowels in Tlingit must be either low or high tone and either long or short.
Roots do not provide either kind of information so they cannot be pronounced alone.
When we speak a root like \fm{\rt[²]{ta}} aloud we actually add tone and length to it, neither of which is part of the root itself.
We can tell that the tone and length are not part of the root because they change depending on the particular form of the verb word.
Consider the examples in (\ref{exx:root+stem-sgsleep}) which show the same root in verbs with a few different 
Each of these has the pronunciation given in square brackets on the second line of the example using the IPA for transcription.

\pex\label{exx:root+stem-sgsleep}%
\a\label{ex:root+stem-sgsleep-impfv}%
\exrtcmt{imperfective aspect}%
\begingl
	\gla	\rlap{\gm{Tá}.} @ {} //
	\glp	\llap{[}\rlap{\ipa{\gm{tʰá}}]} {} //
	\glb	\gm{\rt[¹]{ta}} -μH //
	\glc	\gm{\rt[¹]{sleep·\xx{sg}}} -\xx{var} //
	\gld	\rlap{\xx{impfv}.sleep·\xx{sg}} {} //
	\glft	‘S/he is sleeping.’
		//
\endgl
\a\label{ex:root+stem-sgsleep-pfv}%
\exrtcmt{perfective aspect}%
\begingl
	\gla	\rlap{Woo\gm{taa}.} @ {} @ {} @ {} //
	\glp	\llap{[}\rlap{\ipa{wùː.ˈ\gm{tʰàː}}]} {} {} {} //
	\glb	wu- i- \gm{\rt[¹]{ta}} -μμL //
	\glc	\xx{pfv}- \xx{stv}- \gm{\rt[¹]{sleep·\xx{sg}}} -\xx{var} //
	\gld	\rlap{\xx{pfv}.sleep·\xx{sg}} {} {} {} //
	\glft	‘S/he slept.’
		//
\endgl
\a\label{ex:root+stem-sgsleep-prsp}%
\exrtcmt{prospective aspect}%
\begingl
	\gla	\rlap{Gug̱a\gm{táa}.} @ {} @ {} @ {} @ {} //
	\glp	\llap{[}\rlap{\ipa{kʷùː.qà.ˈ\gm{tʰáː}}]} {} {} {} {} //
	\glb	g- w- g̱- \gm{\rt[¹]{ta}} -μμH //
	\glc	\xx{gcnj}- \xx{irr}- \xx{mod}- \gm{\rt[¹]{sleep·\xx{sg}}} -\xx{var} //
	\gld	\rlap{\xx{prosp}.sleep·\xx{sg}} {} {} {} {} //
	\glft	‘S/he will sleep.’
		//
\endgl
\xe

The verb forms in (\ref{exx:root+stem-sgsleep}) are all based on the same root \fm{\rt[¹]{ta}} ‘sg.\ sleep’.
The verb form in (\ref{ex:root+stem-sgsleep-impfv}) is an imperfective aspect form.
It consists of a single syllable with the consonant /\ipa{tʰ}/, the vowel /\ipa{a}/, short length /\ipa{μ}/, and high tone /\ipa{H}/.
These add together to give [\ipa{tʰá}] as shown below in (\ref{eх:root+stem-sgsleep-phon-shorthigh}).
This is the absolute minimum verb form that can be pronounced and that has an interpretable meaning.

\ex\label{eх:root+stem-sgsleep-phon-shorthigh}%
\exrtcmt{syllable 1 of (\ref{ex:root+stem-sgsleep-impfv})}%
	/\ipa{tʰ}/ + /\ipa{a}/ + /\ipa{μ}/ + /\ipa{H}/
	→ [\ipa{tʰá}]
\xe

The verb form in (\ref{ex:root+stem-sgsleep-pfv}) is a perfective aspect form.
It consists of two syllables, the first of which is formed from the consonant /\ipa{w}/, the vowel /\ipa{u}/, long length /\ipa{μμ}/, and low tone /\ipa{L}/ to give the syllable [\ipa{wùː}] as shown in (\ref{eх:root+stem-sgsleep-phon-longlow-woo}) below.
The second syllable in (\ref{ex:root+stem-sgsleep-pfv}) is formed from the consonant /\ipa{tʰ}/, the vowel /\ipa{a}/, long length /\ipa{μμ}/, and low tone /\ipa{L}/ to give the syllable [\ipa{tʰàː}] as shown in (\ref{eх:root+stem-sgsleep-phon-longlow-taa}).
The syllables [\ipa{wùː}] and [\ipa{tʰàː}] taken together form the word [\ipa{wùː.ˈtʰàː}], with primary stress [\ipa{ˈ}] assigned to the second syllable.

\pex\label{eхx:root+stem-sgsleep-phon-longlow}%
\a\label{eх:root+stem-sgsleep-phon-longlow-woo}%
\exrtcmt{syllable 1 of (\ref{ex:root+stem-sgsleep-pfv})}%
	/\ipa{w}/ + /\ipa{u}/ + /\ipa{μμ}/ + /\ipa{L}/
	→ [\ipa{wùː}]
\a\label{eх:root+stem-sgsleep-phon-longlow-taa}%
\exrtcmt{syllable 2 of (\ref{ex:root+stem-sgsleep-pfv})}%
	/\ipa{tʰ}/ + /\ipa{a}/ + /\ipa{μμ}/ + /\ipa{L}/
	→ [\ipa{tʰàː}]
\xe

The verb form in (\ref{ex:root+stem-sgsleep-prsp}) is a prospective aspect form.
It consists of three syllables, the first of which is formed from the consonant /\ipa{k}/, the vowel /\ipa{u}/, short length /\ipa{μ}/, and low tone /\ipa{L}/ to give the syllable [\ipa{kʷù}] as shown in (\ref{eх:root+stem-sgsleep-phon-longhigh-gu}).
The second syllable is formed from the consonant /\ipa{q}/, the vowel /\ipa{a}/, short length /\ipa{μ}/, and low tone /\ipa{L}/ to give the syllable [\ipa{qà}] as shown in (\ref{eх:root+stem-sgsleep-phon-longhigh-gha}).
The third syllable is formed from the consonant /\ipa{tʰ}/, the vowel /\ipa{a}/, long length /\ipa{μμ}/, and high tone /\ipa{H}/ as shown in (\ref{eх:root+stem-sgsleep-phon-longhigh-taa}).
Taken together these three syllalbes [\ipa{kʷù}], [\ipa{qà}], and [\ipa{tʰáː}] form the word [\ipa{kʷù.qà.ˈtʰáː}] with primary stress [\ipa{ˈ}] again assigned to the last syllable.

\pex\label{eхx:root+stem-sgsleep-phon-longhigh}%
\a\label{eх:root+stem-sgsleep-phon-longhigh-gu}%
\exrtcmt{syllable 1 of (\ref{ex:root+stem-sgsleep-prsp})}%
	/\ipa{k}/ + /\ipa{u}/ + /\ipa{μ}/ + /\ipa{L}/
	→ [\ipa{kʷù}]
\a\label{eх:root+stem-sgsleep-phon-longhigh-gha}%
\exrtcmt{syllable 2 of (\ref{ex:root+stem-sgsleep-prsp})}%
	/\ipa{q}/ + /\ipa{a}/ + /\ipa{μ}/ + /\ipa{L}/
	→ [\ipa{qà}]
\a\label{eх:root+stem-sgsleep-phon-longhigh-taa}%
\exrtcmt{syllable 3 of (\ref{ex:root+stem-sgsleep-prsp})}%
	/\ipa{tʰ}/ + /\ipa{a}/ + /\ipa{μμ}/ + /\ipa{H}/
	→ [\ipa{tʰáː}]
\xe

All three phonological word building patterns in (\ref{eх:root+stem-sgsleep-phon-shorthigh}), (\ref{eхx:root+stem-sgsleep-phon-longlow}), and (\ref{eхx:root+stem-sgsleep-phon-longhigh}) have a syllable composed of /\ipa{tʰ}/ and /\ipa{a}/ which comes from the root \fm{\rt[¹]{ta}} ‘sg.\ sleep’.
They are not identical however: (\ref{eх:root+stem-sgsleep-phon-shorthigh}) is a short vowel with high tone, (\ref{eх:root+stem-sgsleep-phon-longlow-taa}) is a long vowel with low tone, and (\ref{eх:root+stem-sgsleep-phon-longhigh-taa}) is a long vowel with high tone.
Although they have the same consonants and vowels, they have different patterns of length and tone.
These different patterns of length and tone with the same root lead us to identify another unit beyond the root, namely the stem.

\section{Stems and stem variation}\label{sec:root+stem-stems}

The \term{stem} is the part of a verb word which contains the sounds specified by the root together with vowel length and tone.
The stem [\ipa{tʰá}] in (\ref{ex:root+stem-sgsleep-impfv}) is the whole word, but the stems [\ipa{tʰàː}] in (\ref{ex:root+stem-sgsleep-pfv}) and [\ipa{tʰáː}] in (\ref{ex:root+stem-sgsleep-prsp}) are only one syllable within each of their words.
A verb word can therefore be just a stem, but it can also be more than just a stem.

Every verb word must have a stem as illustrated  by the two sentences in (\ref{exx:root+stem-nostem}).
The grammatically possible form in (\ref{ex:root+stem-nostem-withstem}) has its stem highlighted.
The form in (\ref{ex:root+stem-nostem-withoutstem}) is not grammatically possible as indicated by * in front of it. The difference between (\ref{ex:root+stem-nostem-withstem}) and (\ref{ex:root+stem-nostem-withoutstem}) is that the latter does not have a stem, and it is this lack of a stem that makes it ungrammatical.

\pex\label{exx:root+stem-nostem}%
\a\label{ex:root+stem-nostem-withstem}%
\exrtcmt{prospective aspect with stem}%
\begingl
	\gla	\rlap{Gug̱a\gm{táa}.} @ {} @ {} @ {} @ {} //
	\glp	\llap{[}\rlap{\ipa{kʷùː.qà.ˈ\gm{tʰàː}}]} {} {} {} {} //
	\glb	g- w- g̱- \gm{\rt[¹]{ta}} \gm{-μμH} //
	\glc	\xx{gcnj}- \xx{irr}- \xx{mod}- \gm{\rt[¹]{sleep·\xx{sg}}} \gm{-\xx{var}} //
	\gld	\rlap{\xx{prosp}.sleep·\xx{sg}} {} {} {} {} //
	\glft	‘S/he will sleep.’
		//
\endgl
\a\label{ex:root+stem-nostem-withoutstem}%
\ljudge{*}%
\exrtcmt{*prospective aspect without stem}%
\begingl
	\gla	\rlap{Gug̱a.} @ {} @ {} //
	\glp	\llap{[}\rlap{\ipa{kʷùː.qà}]} {} {} //
	\glb	g- w- g̱- //
	\glc	\xx{gcnj}- \xx{irr}- \xx{mod}- //
	\gld	\rlap{\xx{prosp}} {} {} //
	\glft	intended: ‘S/he will.’
		//
\endgl
\xe

From a Tlingit speaking perspective, wanting to say (\ref{ex:root+stem-nostem-withoutstem}) might seem ridiculous.
But the English translation ‘S/he will’ in (\ref{ex:root+stem-nostem-withoutstem}) is fully acceptable and could be used to answer a question like ‘Will s/he sleep?’.
So the idea of speaking just the aspect and tense part of a sentence is entirely plausible, at least in some languages.
It happens to be impossible in Tlingit because verb words in Tlingit require the presence of a verb stem.

The labels ‘root’ and ‘stem’ are standard terms in linguistics for describing the basic formation of words.
These terms are used because they call to mind a metaphor of words as trees.
When a tree grows from seed it puts out two extensions: a root and a stem.
The root extends down into the soil where it is invisible to us on the surface.
The root in a Tlingit verb is similarly invisible because it exists only in the mind of the speaker.
The stem extends up from the soil where we can see it.
The stem in a Tlingit verb is similarly visible because unlike the root it can be spoken aloud.
We can call a root \fm{a x̱aadí} ‘its root’ in Tlingit, and likewise we can call a stem \fm{a kadíx̱ʼi} ‘its stem (plant), its heartwood (tree)’.

As noted earlier, the verb words in (\ref{exx:root+stem-sgsleep}) all have different stems even though they are based on the same root \fm{\rt[¹]{ta}} ‘sg.\ sleep’.
If we return to the syllable building rules in (\ref{eх:root+stem-sgsleep-phon-shorthigh}), (\ref{eх:root+stem-sgsleep-phon-longlow-taa}), and (\ref{eх:root+stem-sgsleep-phon-longhigh-taa}), we can see in detail how each stem can be decomposed.
For convenience, these three rules are repeated together in (\ref{eхx:root+stem-sgsleep-stemphon}) below.
The first two parts of each rule in (\ref{eхx:root+stem-sgsleep-stemphon}) are the same: the consonant /\ipa{tʰ}/ and the vowel /\ipa{a}/.
These are the sounds specified by the root \fm{\rt[¹]{ta}}.
The second two parts of each rule are different: in (\ref{eх:root+stem-sgsleep-stemphon-shorthigh}) they are short length /\ipa{μ}/ and high tone /\ipa{H}/, in (\ref{eх:root+stem-sgsleep-stemphon-longlow}) they are long length /\ipa{μμ}/ and low tone /\ipa{L}/, and in (\ref{eх:root+stem-sgsleep-stemphon-longhigh}) they are long length /\ipa{μμ}/ and high tone /\ipa{H}/.

\pex\label{eхx:root+stem-sgsleep-stemphon}%
\a\label{eх:root+stem-sgsleep-stemphon-shorthigh}%
\exrtcmt{syllable 1 of (\ref{ex:root+stem-sgsleep-prsp})}%
	/\ipa{tʰ}/ + /\ipa{a}/ + /\ipa{μ}/ + /\ipa{H}/
	→ [\ipa{tʰá}]
\a\label{eх:root+stem-sgsleep-stemphon-longlow}%
\exrtcmt{syllable 2 of (\ref{ex:root+stem-sgsleep-pfv})}%
	/\ipa{tʰ}/ + /\ipa{a}/ + /\ipa{μμ}/ + /\ipa{L}/
	→ [\ipa{tʰàː}]
\a\label{eх:root+stem-sgsleep-stemphon-longhigh}%
\exrtcmt{syllable 3 of (\ref{ex:root+stem-sgsleep-prsp})}%
	/\ipa{tʰ}/ + /\ipa{a}/ + /\ipa{μμ}/ + /\ipa{H}/
	→ [\ipa{tʰáː}]
\xe

Having distilled out the root, we can identify \term{stem variation} as the remaining material that varies from stem to stem.
Stem variation is represented as an abstract suffix by combining length and tone symbols.
Thus, looking back at (\ref{ex:root+stem-sgsleep-impfv}) the stem variation – glossed \xx{var} – is the suffix \fm{-μH} for a short vowel with high tone.
Likewise, the stem variation in (\ref{ex:root+stem-sgsleep-pfv}) is the suffix \fm{-μμL} for a long vowel with low tone, and the stem variation in (\ref{ex:root+stem-sgsleep-prsp}) is the suffix \fm{-μμH} for a long vowel with high tone.
Stem variation is dependent on several different grammatical phenomena, most notably including aspect, tense, negation, and clause type.
Some details about the patterns of stem variation will be explored later in chapter \ref{ch:stemvar}.

\begin{figure}
\centerfloat
\begin{tikzpicture}[mytree,
	baseline=(top.center),
	level distance=2em,
	sibling distance=10em,
	text height=0.7em,
	text depth=0.25em]
\node (top) {stem}
	child {node {\rt{\xx{root}}}
		[sibling distance=3em]
		child {node (onset) {C}}
		child {node (nucleus) {V}}
		child {node (coda) {(C)}}}
	child {node {\llap{-}\xx{var}}
		[sibling distance=3em]
		child {node (length) {μ(μ)}}
		child {node (tone) {H/L}}};
\node[below=-0.25em of onset]	{onset};
\node[below=-0.25em of nucleus]	{vowel};
\node[below=-0.25em of coda]	{coda};
\node[below=-0.25em of length]	{length};
\node[below=-0.25em of tone]	{tone};
\end{tikzpicture}
\caption{Diagram of relationships between stems, roots, and stem variation.}
\label{fig:root+stem-root-stem-and-variation}
\end{figure}

Stems are made of roots and stem variation.
The diagram in figure \ref{fig:root+stem-root-stem-and-variation} illustrates the relationship between these three elements of a root, stem variation, and a stem.
The root internally contains an onset consonant /\ipa{C}/, a vowel /\ipa{V}/, and optionally a coda consonant /\ipa{C}/.
The stem variation internally contains length which may be either short /\ipa{μ}/ or long /\ipa{μμ}/ and tone which may be either high /\ipa{H}/ or low /\ipa{L}/.
The root and the stem variation are separate elements that combine to form the stem.

\begin{figure}
\centerfloat
\begin{tikzpicture}
\matrix[matrix of nodes,
	row sep=3em,
	column sep={2em,between origins},
	column 5/.style={anchor=base west}]
{
			& |(v2)| /\ipa{μμ}/	& |(v3)| /\ipa{L}/			&				& |(v5)| stem variation \fm{-μμL}\\
|(s1)| [\ipa{kʷ}	& |(s2)| \ipa{ù}		& |(s3)| \ipa{\vphantom{ù}ː}	& |(s4)| \ipa{t}\rlap{]}	& |(s5)| stem \fm{goot}\\
|(r1)| /\ipa{kʷ}/	& |(r2)| /\ipa{u}/		&						& |(r4)| /\ipa{t}/		& |(r5)| root \fm{\rt{gut}}\\
};
\draw[exarrows] (v2) -- (s2);
\draw[exarrows] (v2) -- (s3.north);
\draw[exarrows] (v3) -- (s2);
\draw[exarrows] (r1) -- (s1);
\draw[exarrows] (r2) -- (s2);
\draw[exarrows] (r4) -- (s4);
\end{tikzpicture}
\caption{Construction of a stem with a CVC root and \fm{-μμL} stem variation.}
\label{fig:root+stem-example-stem-docking}
\end{figure}

The diagram in figure \ref{fig:root+stem-example-stem-docking} illustrates how the root and stem variation are combined together to produce a stem.
The particular stem shown in figure \ref{fig:root+stem-example-stem-docking} is from the verb word \fm{x̱waagoot} ‘I went’ which is analyzed below in (\ref{ex:root+stem-gosg}).
This is a perfective aspect form based on the root \fm{\rt{gut}} ‘singular go’ which has an onset consonant /\ipa{kʷ}/, a vowel /\ipa{u}/, and a coda consonant /\ipa{t}/.
The stem variation in (\ref{ex:root+stem-gosg}) is \fm{-μμL} which is a long vowel /\ipa{μμ}/ and low tone /\ipa{L}/.
The consonants and vowel of the root map directly onto the consonants and the vowel of the stem.
The length and tone of the stem variation map onto the vowel of the stem.

\ex\label{ex:root+stem-gosg}%
\exrtcmt{perfective aspect}%
\begingl
	\gla	{} \rlap{Aadé} @ {} {} \rlap{x̱waa\gm{goot}.} @ {} @ {} @ {} @ {} //
	\glp	{} \llap{[}\rlap{\ipa{ˈʔàː.té}} {} {} \rlap{\ipa{χʷàː.ˈ\gm{kʷùːt}}]} {} {} {} {} //
	\glb	{} á -dé {} wu- x̱- i- \gm{\rt[¹]{gut}} \gm{-μμL} //
	\glc	{}[\pr{PP} \xx{3n} -\xx{all} {}] \xx{pfv}- \xx{1sg·s}- \xx{stv}- \gm{\rt[¹]{go·\xx{sg}}} \gm{-\xx{var}} //
	\gld	{} there -to {} \rlap{\xx{pfv}.I.go·\xx{sg}} {} {} {} {} //
	\glft	‘I went there.’
		//
\endgl
\xe

Most verb words have their stem at the end, so that the stem is the last syllable.
This is true for all of the stems presented so far in this section.
Exceptions arise when the verb word includes suffixes beyond the stem variation suffix.
These moderately common; some typical examples are shown in (\ref{exx:root+stem-sgsleep-suffix}).

\pex\label{exx:root+stem-sgsleep-suffix}%
\a\label{ex:root+stem-sgsleep-suffix-prsp+past}%
\exrtcmt{prospective aspect + past tense}%
\begingl
	\gla	\rlap{Gug̱a\gm{taa}yín.} @ {} @ {} @ {} @ {} @ {} //
	\glp	\llap{[}\rlap{\ipa{kʷùː.qà.ˈ\gm{tʰàː}.jín}]} {} {} {} {} {} //
	\glb	g- w- g̱- \gm{\rt[¹]{ta}} \gm{-μμL} -ín //
	\glc	\xx{gcnj}- \xx{irr}- \xx{mod}- \gm{\rt[¹]{sleep·\xx{sg}}} \gm{-\xx{var}} -\xx{past} //
	\gld	\rlap{\xx{prosp}.sleep·\xx{sg}.\xx{past}} {} {} {} {} {} //
	\glft	‘S/he would have slept.’
		//
\endgl
\a\label{ex:root+stem-sgsleep-suffix-pfv+sub}%
\exrtcmt{pfv.\ + subord.}%
\begingl
	\gla	{} {} \rlap{Wu\gm{taa}yích,} @ {} @ {} @ {} {} {} {}
		{} du \rlap{toowú} @ {} {} \rlap{yakʼéi.} @ {} @ {} //
	\glp	{} {} \llap{[}\rlap{\ipa{wùː.ˈ\gm{tʰàː}.jítʃ}} {} {} {} {} {} {}
		{} \ipa{tù} \rlap{\ipa{ˈtʰùː.wú}} {} {} \rlap{\ipa{jà.ˈkʼéː}]} {} {} //
	\glb	{} {} wu- \gm{\rt[¹]{ta}} \gm{-μμL} -í {} -ch {}
		{} du tú -í {} i- \rt[¹]{kʼe} -μμH //
	\glc	{}[\pr{PP} {}[\pr{CP} \xx{pfv}- \gm{\rt[¹]{sleep·\xx{sg}}} \gm{-\xx{var}} -\xx{sub} {}] -\xx{erg} {}]
		{}[\pr{DP} \xx{3h·pss} inside -\xx{pss} {}] \xx{stv}- \rt[¹]{good} -\xx{var} //
	\gld	{} {} \rlap{\xx{pfv}.sleep·\xx{sg}.\xx{sub}} {} {} {} {} {} {}
		{} h/h \rlap{feelings} {} {} \rlap{\xx{stv}·\xx{impfv}.good} {} {} //
	\glft	‘Because s/he slept, his/her feelings are good.’
		//
\endgl
\xe

The form in (\ref{ex:root+stem-sgsleep-suffix-prsp+past}) has the past tense suffix \fm{-ín} at the end.
This suffix is pronounced as an additional syllable [\ipa{jín}] after the stem so that the stem is no longer the last (final, ultimate) syllable but is instead the next to last (penultimate)%
\footnote{\fm{Penultimate} is from Latin \fm{paene} ‘almost’ + \fm{ultimus} ‘last’.} syllable.
The form in (\ref{ex:root+stem-sgsleep-suffix-pfv+sub}) has the subordinate clause suffix \fm{-í} followed by the ergative suffix \fm{-ch}.
These combine to give another syllable [\ipa{jítʃ}] so that once again the stem is the penultimate syllable.

The most common case where the stem is not the final syllable is certainly that described above where the stem is penultimate.
There are very rare cases where multiple suffixes result in the stem as the next to next to last (antepenultimate)%
\footnote{\fm{Antepenultimate} is from Latin \fm{ante} ‘before’ + \fm{paene} ‘almost’ + \fm{ultimus} ‘last’.} syllable.
One example of this is shown in (\ref{ex:root+stem-ha-suffix+suffix}) where there are three suffixes after the stem: deprivative \fm{-áḵw} ‘lacking’, repetitive \fm{-ch}, and subordinate clause \fm{-í}.
The stem [\ipa{héː}] is followed by the syllables [\ipa{jáqʷ}] and [\ipa{tʃì}].%
\footnote{The change of vowel from /\ipa{a}/ to [\ipa{e}] is an instance of ablaut which is detailed in chapter \ref{ch:stemvar}.}
There are as yet no cases of verb words where the stem is more than two syllables from the end of the word.

\ex\label{ex:root+stem-ha-suffix+suffix}%
\exrtcmt{several suffixes on a verb}%
\begingl
	\gla	\rlap{yakoog̱as.\gm{héi}yáḵwji} @ {} @ {} @ {} @ {} @ {} @ {} @ {} @ {} @ {} @ {} //
	\glp	\llap{[}\rlap{\ipa{jà.kʷʰùː.qàs.ˈ\gm{héː}.jáqʷ.tʃì}]} {} {} {} {} {} {} {} {} {} {} //
	\glb	ÿ- k- w- g̱- d- s- \gm{\rt[¹]{ha}} \gm{-eμH} -áḵw -ch -í //
	\glc	\xx{qual}- \xx{qual}- \xx{irr}- \xx{g̱cnj}- \xx{pasv}- \xx{csv}- \gm{\rt[¹]{mv·invis}}
			\gm{-\xx{var}} -\xx{dep} -\xx{rep} -\xx{sub} //
	\gld	\rlap{\xx{hab}.\xx{pasv}.make.dwindle} {} {} {} {} {} {} {} {} {} {} //
	\glft	‘it would be made to dwindle away’
		\trailingcitation{\parencite[01/19]{leer:1973}}
		//
\endgl
\xe

Every verb word so far has included primary stress [\ipa{ˈ…}] on the verb stem syllable.
This is not an accident.
Although we have not done much research on syllable stress in Tlingit, it is readily apparent just from listening to the language that all verbs have primary stress on the stem syllable.
Because stress is not represented in the orthography, in written Tlingit it is difficult to use stress as an aid to identifying the stem.
But in spoken Tlingit the sound of stress (pitch difference, loudness, length) is fairly easy to hear so it is extremely useful in learning to identify different verbs.

\stopcontents[chapters]