%!TEX root = ../building-verbs.tex
%%
%% Colophon.
%%

%% Note: This file should be loaded with \input rather than \include so that
%% it is always present in the output PDF of every typesetting run.

\chapter{Colophon}

\vspace*{\baselineskip}
\noindent{}This book was written in \LaTeXe\ and a small amount of \LaTeX\ 3,
using \XeLaTeX\ 3.14159265-2.6-0.999991 (\TeX\ Live 2019),
Memoir class 3.7j,
Fontspec 2.7d,
Bib\LaTeX\ 3.13a
	with \textsc{lsa}like 0.8j
	and Biber 2.12,
Hyperref 7.00c with HyperXMP 4.1,
Ti\textit{k}Z/\textsc{\MakeLowercase{PGF}} 3.1.4b,
ExPex 5.1b,
BibDesk 1.7.2 (5037),
and \TeX{}Shop 4.44
on macOS 10.14 ‘Mojave’.
The chapter layout is Memoir’s ‘wilsondob’ style. The page layout is adapted from Memoir’s ‘plain’ style. The table of contents and list of \{tables, figures\} layout is based on Memoir’s default style with minor changes.

\vspace*{\baselineskip}
\noindent{}The primary typeface is Brill 2.06 (build 051) by John Hudson of Tiro Typeworks (\url{http://www.tiro.com}), designed for Koninklijke Brill NV (\url{http://www.brill.com}) and made freely available for non-commercial use. Sans-serif text is set in \textsf{Calibri 5.72} and monospace text in \texttt{Consolas} \texttt{5.33}, both by Luc(\:as\:) de Groot for Microsoft Corporation. Many mathematical symbols are set in {\mathemafont Asana} {\mathemafont Math} {\mathemafont 000.954} by Apostolos Syropoulos, Young Ryu, and Claudio Beccari which is based on
%{\palafont Palatino}
Palatino
by Hermann Zapf. Certain symbols are set in {\ckfont Lucida} {\ckfont Grande} {\ckfont 10.0d1e2} by Kris Holmes and Charles Bigelow or in {\symfont Apple} {\symfont Symbols} {\symfont 10.0d1e2} by Apple Computer, Inc. No sheep were stolen in the typesetting of this document.

\vspace*{\baselineskip}
\noindent{}Source code for this book is available from \url{https://github.com/jcrippen/building-verbs}.

\vspace*{\baselineskip}
\noindent{}This \textsc{\MakeLowercase{PDF}} was typeset by the author on
	\today\
	at
	\printtime\
	\textsc{\MakeLowercase{PST}} (\:−0800\:).